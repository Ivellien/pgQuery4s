\section{Scala}
\subsection{Introduction}
Scala belongs to the group of programming languages that can be compiled into Java byte code and run on a Java virtual machine (JVM). The major part, which makes it different from well-known Java, is the combination of applying a functional approach with an object-oriented paradigm. Together with the fact that Scala is similar to Java language itself, having the object-oriented style still present can ease up transition for programmers who are unfamiliar with the functional world.

\subsection{Static typing}

Besides the functional fundamentals, Scala belongs to the family of statically typed languages. This family also includes languages like C, C++, Java, or Haskell. Therefore, every single statement in Scala has a type.\cite{Scala static}

To make a job easier for the programmer, Scala uses a system known as type inference - automatic type detection. That allows faster coding, thanks to the fact you don’t have to worry about specifying every object’s type. 

Since Scala compiler knows the types of every statement, it is able to reveal bugs during compilation. That is a great thing, because the sooner we can identify a bug, the easier it should be to fix it. 
