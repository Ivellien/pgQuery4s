\section{Parse tree representation}
\subsection{C}

If we look at internal representation of the tree directly in the PostgreSQL parser, it defines each possible node of the parsetree in form of \texttt{struct}. Any type of node is guaranteed to have \texttt{NodeTag} as the first field. 
\begin{itemize}
    \item \texttt{enum NodeTag} - defines all possible types of nodes
    \item \texttt{struct Node} - contains only one variable - the \texttt{NodeTag}
\end{itemize}
Thanks to that guarantee, any type of node can be cast to \texttt{Node} without losing the information about the type of the node. That allows casting the \texttt{Node} back to the original type when needed.
This fact is also used when a node contains another node as a leaf. Most of the time, the type of the required node isn't specified directly, instead \texttt{Node} pointer is used. This provides flexibility for the parameters of the nodes. A great example of that is the node of type \texttt{A\_Expr}, where both left side expression and right side expression have \texttt{Node*} type.
\begin{lstlisting}[language=C, basicstyle=\ttfamily]      
typedef struct A_Expr {
    NodeTag     type;
    A_Expr_Kind kind;
    List       *name; 
    Node       *lexpr; 
    Node       *rexpr; 
    int         location; 
} A_Expr;
\end{lstlisting}


\subsection{Scala}
Since we already have existing parse tree representation in C we can mirror it in Scala.
Scala is an object-oriented language, therefore each type of node should be defined by its own class. However, using case classes offers several advantages over classes in Scala: 
\begin{itemize}
    \item Case classes can be pattern matched
    \item Automatic definition of equals and hashcode methods
    \item Automatic definition of getters
\end{itemize}
First, we convert the \texttt{struct Node} into \texttt{abstract class Node}. It will be used as base class for every node of the parse tree. \\
Each \texttt{struct} can be then converted into case class in Scala. 
\begin{itemize}
    \item Each \texttt{Node*} is converted to \texttt{Option[Node]} - same for other variables, which are pointers to specific \texttt{Node} type
    \item Each \texttt{List*} is converted to \texttt{List[Node]}
    \item Primitive data types are converted to their Scala equivalents \\
    (e.g. \texttt{int} to \texttt{Int}, \texttt{char*} to \texttt{String})
    \item The NodeTag parameter can be ommited, because in Scala we can use pattern matching to check the type of the \texttt{Node}.
\end{itemize}
Each \texttt{enum} is converted to object extending abstract class Enumeration.
 \\
\textbf{Converted \texttt{A\_Expr} case class}
\begin{lstlisting}[language=scala, basicstyle=\ttfamily]
case class A_Expr(
    kind:     A_Expr_Kind.Value,
    name:     List[Node],
    lexpr:    Option[Node],
    rexpr:    Option[Node],
    location: Int
) extends Node
\end{lstlisting}