\section{Scala custom interpolators}
\subsection{What are interpolators?}
\textit{Starting in Scala 2.10.0, Scala offers a new mechanism to create strings from your data: String Interpolation. String Interpolation allows users to embed variable references directly in processed string literals.}~\cite{String interpolation} Besides the three interpolators provided by scala, we can also define our own custom interpolator. This would help us to create generic queries with variables instead of direct values. That way, we can define and reuse queries without unnecessary copying and pasting of code. By extending the existing \texttt{StringContext} class with new method, we can introduce custom interpolators. 
\bigskip
\newline
\begin{lstlisting}[language=scala, basicstyle=\ttfamily, showstringspaces=false, caption={Example of \texttt{String} concatenation}]
val query: String = 
 "SELECT " + columnName + " FROM students"
PgQueryParser.parse(query)
\end{lstlisting}
\bigskip
\begin{lstlisting}[language=scala, basicstyle=\ttfamily, showstringspaces=false, caption={Example of custom interpolation}]
query"SELECT $columnName FROM students"
\end{lstlisting}

\subsection{Runtime implementation}
Although the goal of this project is to validate queries during compilation and transform the interpolated string to \texttt{Node} at compile time, the runtime validation is important as well. Parse trees of queries are accessible at runtime. Simply use the built-in string interpolator to create the query. The parse tree can be generated by the \texttt{parse} method of \texttt{PgQueryParser}.