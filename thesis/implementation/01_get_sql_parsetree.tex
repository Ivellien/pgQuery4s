
\section{Getting the SQL parse tree}
Since we want our library to work with parse trees of the SQL queries, the first question we have to ask is how to get the internal parse tree itself. According to the official documentation\cite{PostgreSQL documentation}, the parse tree gets created during the "Parser stage". Fortunately, there already exists a project, which extracts the parse tree called \texttt{Libpg\_query}.

\subsection{Libpg\_query}
\texttt{Libpg\_query} is an open-source C library created by Lukas Frittl. It uses parts of the PostgreSQL server to access the internal \texttt{raw\_parse} function, which returns the internal parse tree. A minor disadvantage of this approach is that it uses the server code directly and it has to be compiled before it can be used. Then it accesses internal functions of the server which allows the library to get the parse tree for each valid query.

The main purpose of \texttt{libpg\_query} is to be used as a base library for implementations in other languages. There already exist multiple wrappers, for example \texttt{pg\_query} for Ruby or \texttt{pglast} for Python. However, at the moment of writing this thesis, there is no existing wrapper for it written for Scala. The important function from \texttt{libpg\_query} is the \texttt{pg\_query\_parse} function.

The \texttt{pg\_query\_parse} takes the plain text SQL query in form of \texttt{const char*}. Then it calls the extracted parts of the PostgreSQL server and returns the parse tree as JSON. Once we have that, we can decode the JSON and map it onto the created case class structure in Scala.

\newpage
Here we have a simple code snippet describing simple usage taken from the GitHub README\cite{libpgquery}. 

\begin{lstlisting}[language=C, basicstyle=\ttfamily, keywordstyle=\color{purple}]
#include <pg_query.h>
#include <stdio.h>

int main() {
  PgQueryParseResult result;
  result = pg_query_parse("SELECT 1");
  printf("%s\n", result.parse_tree);
  pg_query_free_parse_result(result);
}
\end{lstlisting}
