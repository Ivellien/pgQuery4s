\section{Parse tree representation in Scala}
Before we can start our work with the parse tree, it will prove useful to create our own structure, to represent that data in a form we can easily work with. 

The C library has its own struct representation for each type of possible Nodes that can be found in the internal PostgreSQL parse tree. That makes our job easier, because we simply have to transform C structs to Scala case classes.

\bigskip
\textbf{Original C struct in \texttt{libpg\_query}}
\begin{lstlisting}[language=C, basicstyle=\ttfamily]      
typedef struct A_Expr {
    NodeTag     type;
    A_Expr_Kind kind;
    List       *name; 
    Node       *lexpr; 
    Node       *rexpr; 
    int         location; 
} A_Expr;
\end{lstlisting}

\textbf{Scala case class}
\begin{lstlisting}[language=scala, basicstyle=\ttfamily]
case class A_Expr(
    kind:     A_Expr_Kind.Value,
    name:     List[Node],
    lexpr:    Option[Node],
    rexpr:    Option[Node],
    location: Option[Int]
) extends Node
\end{lstlisting}