% arara: xelatex
% arara: xelatex
% arara: xelatex


% options:
% thesis=B bachelor's thesis
% thesis=M master's thesis
% czech thesis in Czech language
% english thesis in English language
% hidelinks remove colour boxes around hyperlinks

\documentclass[thesis=B,english]{FITthesis}[2019/12/23]

%\usepackage[utf8]{inputenc} % LaTeX source encoded as UTF-8
% \usepackage[latin2]{inputenc} % LaTeX source encoded as ISO-8859-2
% \usepackage[cp1250]{inputenc} % LaTeX source encoded as Windows-1250

% \usepackage{subfig} %subfigures
% \usepackage{amsmath} %advanced maths
% \usepackage{amssymb} %additional math symbols

\usepackage{dirtree} %directory tree visualisation
\usepackage{listings}
\usepackage{subfiles}

% % list of acronyms
% \usepackage[acronym,nonumberlist,toc,numberedsection=autolabel]{glossaries}
% \iflanguage{czech}{\renewcommand*{\acronymname}{Seznam pou{\v z}it{\' y}ch zkratek}}{}
% \makeglossaries

% % % % % % % % % % % % % % % % % % % % % % % % % % % % % % 
% EDIT THIS
% % % % % % % % % % % % % % % % % % % % % % % % % % % % % % 

\department{Department of software engineering}
\title{Scala library for constructing statically typed PostgreSQL queries}
\authorGN{Petr} %author's given name/names
\authorFN{Hron} %author's surname
\author{Petr Hron} %author's name without academic degrees
\authorWithDegrees{Petr Hron} %author's name with academic degrees
\supervisor{Ing. Vojtěch Létal}
\acknowledgements{THANKS (remove entirely in case you do not with to thank anyone)}
\abstractEN{Summarize the contents and contribution of your work in a few sentences in English language.}
\abstractCS{V n{\v e}kolika v{\v e}t{\' a}ch shr{\v n}te obsah a p{\v r}{\' i}nos t{\' e}to pr{\' a}ce v {\v c}esk{\' e}m jazyce.}
\placeForDeclarationOfAuthenticity{Prague}
\keywordsCS{Scala, PostgreSQL, syntaktický strom, open source, validace během kompilace}
\keywordsEN{Scala, PostgreSQL, parse tree, open source, compile time validation}
\declarationOfAuthenticityOption{1} %select as appropriate, according to the desired license (integer 1-6)
% \website{http://site.example/thesis} %optional thesis URL


\begin{document}

% \newacronym{CVUT}{{\v C}VUT}{{\v C}esk{\' e} vysok{\' e} u{\v c}en{\' i} technick{\' e} v Praze}
% \newacronym{FIT}{FIT}{Fakulta informa{\v c}n{\' i}ch technologi{\' i}}

\setsecnumdepth{part}
\setsecnumdepth{all}
\chapter{Introduction}

\subfile{introduction/motivation}

\chapter{Technologies used}

\subfile{introduction/scala}

\subfile{introduction/postgresql}

\chapter{Existing options}

\subfile{background/database_libs}

\chapter{Realisation}

\subfile{implementation/get_sql_parsetree}

\subfile{implementation/parsetree_representation}

\subfile{implementation/native_libary}

\subfile{implementation/parsing_json}

\subfile{implementation/interpolators}

\subfile{implementation/macros}

\subfile{implementation/combine_macro_interpolator}

\section{Testing}
\section{Summary}

\section{Publishing library}

\chapter{Conclusion}


\bibliographystyle{iso690}
\bibliography{mybibliographyfile}
\begin{thebibliography}{9}

\bibitem{PostgreSQL}
\textit {What Is PostgreSQL?} [online]. [cit. 2021-06-21]. Available from:
https://www.postgresqltutorial.com/what-is-postgresql/

\bibitem{Stackoverflow survey}
\textit {Stack Overflow Annual Developer Survey} [online]. [cit. 2021-06-13]. Available from: https://insights.stackoverflow.com/survey/2020\#technology-databases-all-respondents4

\bibitem{Scala static}
Mayank Bhatnagar
\textit {Magic lies here - Statically vs Dynamically Typed Languages} [online]. [cit. 2021-06-21]. Available from: https://medium.com/android-news/magic-lies-here-statically-typed-vs-dynamically-typed-languages-d151c7f95e2b

\bibitem{Quill} 
\textit {What is Quill?} [online]. [cit. 2021-04-25]. Available from: https://github.com/getquill/quill/
\bibitem{Doobie}
NORRIS, Rob. 
\textit {Doobie documentation} [online]. [cit. 2021-04-25]. Available from: https://tpolecat.github.io/doobie/
\bibitem{PostgreSQL documentation}
The PostgreSQL Global Development Group
\textit{The Parser Stage} [online]. [cit. 2021-06-20]. Available from: https://www.postgresql.org/docs/10/parser-stage.html
\bibitem{libpgquery}
FITTL, Lukas.
\textit {libpg\_query} [online]. [cit. 2021-04-25]. Available from:
https://github.com/pganalyze/libpg\_query

\bibitem{sbt fork}
\textit {Forking} [online]. [cit. 2021-06-22]. Available from:
https://www.scala-sbt.org/0.12.3/docs/Detailed-Topics/Forking.html

\bibitem{JNI}
Oracle
\textit{Java Native Interface} [online]. [cit. 2021-06-20]. Available from:
https://docs.oracle.com/javase/8/docs/technotes/guides/jni/

\bibitem{String interpolation}
SUERETH, Josh. 
\textit {String interpolation} [online]. [cit. 2021-04-25]. Available from: https://docs.scala-lang.org/overviews/core/string-interpolation.html
\bibitem{Def macros}
BURMAKO, Eugene. 
\textit {Def macros} [online]. [cit. 2021-04-25]. Available from: https://docs.scala-lang.org/overviews/macros/overview.html
\bibitem{Compile-time reflection}
MILLER, Heather, BURMAKO Eugene and HALLER Philipp 
\textit {Compile-time reflection} [online]. [cit. 2021-06-19]. Available from: https://docs.scala-lang.org/overviews/reflection/overview.html\#compile-time-reflection

\end{thebibliography}
\setsecnumdepth{all}
\appendix

\chapter{Acronyms}
% \printglossaries
\begin{description}
	\item[API] Application programming interface
	\item[JDBC] Java Database Connectivity
	\item[JPA] Jakarta Persistence
	\item[JSON] JavaScript Object Notation
	\item[JVM] Java virtual machine
	\item[SQL] Structured Query Language
\end{description}


\chapter{Contents of enclosed CD}

%change appropriately

\begin{figure}
	\dirtree{%
		.1 readme.txt\DTcomment{the file with CD contents description}.
		.1 exe\DTcomment{the directory with executables}.
		.1 src\DTcomment{the directory of source codes}.
		.2 wbdcm\DTcomment{implementation sources}.
		.2 thesis\DTcomment{the directory of \LaTeX{} source codes of the thesis}.
		.1 text\DTcomment{the thesis text directory}.
		.2 thesis.pdf\DTcomment{the thesis text in PDF format}.
		.2 thesis.ps\DTcomment{the thesis text in PS format}.
	}
\end{figure}

\end{document}
